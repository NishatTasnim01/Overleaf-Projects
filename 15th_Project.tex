\documentclass{article}
\usepackage[utf8]{inputenc}
\usepackage{amsthm}

\newtheorem{theorem}{\textbf{Theorem}}
\newtheorem{definition}{\textbf{Definition}}
\newtheorem{lemma}{\textbf{Lemma}}
\newtheorem{corollary}{\textbf{Corollary}}[theorem]
%\newtheorem{corollary}[theorem]{Corollary}

\begin{document}

\section{Mathematical Statements:}

\begin{theorem}[Pythagoras]\label{pythagoras}
    The square of the hypotenuse of a right triangle is equal to the sum of the squares of the other two sides.
\end{theorem}

\begin{corollary}
    In a right triangle, the square of the length of the hypotenuse (c) is equal to the sum of the squares of the lengths of the other two sides (a and b):
    \[ c^2 = a^2 + b^2 \]
\end{corollary}

\begin{proof}
    Let \(a\), \(b\), and \(c\) be the sides of a right triangle as described in Theorem~\ref{pythagoras}. The proof follows by the application of the Pythagorean theorem.
\end{proof}

\begin{definition}
    A right triangle is a triangle in which one angle is a right angle (90 degrees).
\end{definition}

\begin{definition}
    The hypotenuse of a right triangle is the side opposite the right angle.
\end{definition}

\begin{lemma}
    In the context of Theorem~\ref{pythagoras}, the square of the hypotenuse is equal to the sum of the squares of the other two sides.
\end{lemma}

\section{Advanced Mathematics}

\subsection{Number Theory}

\begin{theorem}[Fermat's Last Theorem]\label{fermat}
    No three positive integers \(a\), \(b\), and \(c\) satisfy the equation \(a^n + b^n = c^n\) for any integer value of \(n\) greater than 2.
\end{theorem}

\begin{corollary}
    Fermat's Last Theorem implies that there are no whole number solutions to the equation \(x^3 + y^3 = z^3\).
\end{corollary}

\begin{proof}
    The proof of Fermat's Last Theorem is complex and was not discovered until centuries after Fermat's death. It involves advanced mathematical concepts and is beyond the scope of this document.
\end{proof}

\subsection{Calculus}

\begin{theorem}[Fundamental Theorem of Calculus]\label{ftc}
    If \(f(x)\) is continuous on the interval \([a, b]\) and \(F(x)\) is an antiderivative of \(f(x)\), then
    \[ \int_{a}^{b} f(x) \,dx = F(b) - F(a) \]
\end{theorem}

\begin{corollary}
    The definite integral of a function \(f(x)\) over an interval \([a, b]\) can be computed by finding the antiderivative \(F(x)\) and evaluating it at the endpoints.
\end{corollary}

\begin{proof}
    The proof involves the properties of antiderivatives and the concept of the accumulation of values over an interval.
\end{proof}

\end{document}