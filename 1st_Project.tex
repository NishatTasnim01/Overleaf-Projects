\documentclass{article}
\usepackage{graphicx} % Required for inserting images
\usepackage{verbatim}

\title{Programming with Java Language}
\author{Nishat Tasnim}
\date{December 2023}

\begin{document}

\maketitle
\vspace{-5mm}
\begin{center}
    % My name is Nishat Tasnim. I'm a student.
\end{center}

\section{Introduction}
In this section, we provide an overview of Java programming, highlighting its key features and applications.Java is a versatile and widely-used programming language known for its portability, performance, and security features.\\ Whether you are a beginner or an experienced developer, understanding Java fundamentals is essential for building robust software applications. Let's explore the basics of Java programming together.

\begin{comment}
    In this section, we provide an overview of Java programming, highlighting its key features and applications. Java is a versatile and widely-used programming language known for its portability, performance, and security features. Whether you are a beginner or an experienced developer, understanding Java fundamentals is essential for building robust software applications. Let's explore the basics of Java programming together.
\end{comment}

\section{Data Types}
Java supports a variety of data types, which can be broadly categorized into primitive and non-primitive types.

\subsection{Primitive Data Types}
Primitive data types are the building blocks of Java programs.

\subsubsection{Integer}
The integer data type represents whole numbers without any decimal points.

\subsubsection{String}
The string data type is used to represent sequences of characters.

\subsection{Non-Primitive Data Types}
Non-primitive data types include more complex structures, such as arrays and objects.

\section{Operators}
Operators in Java are symbols that perform operations on variables and values.

\section{Control Flow Statements}
Control flow statements manage the flow of execution in a Java program.

\subsection{Conditional Statements}
Conditional statements alter the flow of execution based on a condition.

\subsubsection{if Statement}
The `if` statement allows the program to make decisions based on a condition.

\subsubsection{switch Statement}
The `switch` statement is used for multi-way branching based on different cases.

\subsection{Looping Statements}
Looping statements repeat a block of code multiple times.

\subsubsection{for Loop}
The `for` loop is used to iterate a block of code a specific number of times.

\subsubsection{while Loop}
The `while` loop repeats a block of code while a given condition is true.

\end{document}
