\documentclass{article}
\usepackage{graphicx} % Required for inserting images

\title{Java Programming Basics}
\author{Nishat Tasnim}
\date{December 2023}

\begin{document}

\maketitle

\section{Introduction}
\tiny
Welcome to the exciting realm of Java programming!\\
\small
Within these pages, our goal is to furnish you with a thorough understanding of the fundamental concepts and distinctive features that define the Java programming language.\\
\footnotesize
As you delve into this document, you'll gain insights into the core elements that make Java a powerful and versatile programming tool.\\
\large
Whether you are a novice or an experienced developer, this comprehensive overview will serve as a valuable resource on your journey through Java.\\
\huge
Explore the intricacies of Java and unlock the potential to create robust and sophisticated applications that propel you to new heights in the world of programming.

\section{Data Types}
\large
\textbf{Java offers a diverse range of data types to handle various kinds of values.}

\small
\textit{\\We categorize them into two primary groups:}

\large
\underline{Primitive data types and Non-primitive data types.}

\subsection{Primitive Data Types}
\large
\textbf{\textit{Primitive data types serve as the fundamental building blocks for representing simple values. Two commonly used primitive data types are:}}

\subsubsection{Integer}
Integers represent whole numbers without any decimal points. Examples include positive integers like 1, negative integers like -5, and large numbers like 100.

\subsubsection{String}
Strings are sequences of characters. They play a crucial role in representing textual data, such as names, sentences, and alphanumeric combinations.

\subsection{Floating-point}
Floating-point numbers are used to represent real numbers with decimal points. Examples include constants like 3.14, negative values like -0.5, and whole numbers with decimal points like 2.0.

\subsection{Boolean}
The Boolean data type represents true or false values. It is frequently employed in conditions and logical expressions to control the flow of a program.

\subsection{Character}
The character data type is used to represent individual characters, such as letters, digits, or special symbols like '$\%$'.

\subsection{Non-Primitive Data Types}
Non-primitive data types, also known as reference types, are more sophisticated and can represent a combination of values or objects.

\subsubsection{Array}
Arrays enable the storage of multiple values of the same type in a single variable, providing a convenient way to manage and manipulate data.

\subsubsection{Class}
Classes serve as the foundational units of object-oriented programming in Java. They encapsulate both data and behavior, facilitating modular and reusable code.

\section{Operators}
Operators in Java are symbolic entities that perform operations on variables or values, allowing for the manipulation of data and the execution of calculations.

\subsection{Arithmetic Operators}
Arithmetic operators carry out fundamental mathematical operations, including addition, subtraction, multiplication, and division.

\subsection{Comparison Operators}
Comparison operators facilitate the comparison of values, yielding a Boolean result that indicates whether the comparison is true or false.

\subsection{Logical Operators}
Logical operators perform logical operations such as AND, OR, and NOT. They are integral in constructing conditional statements and controlling program flow.

\section{Control Flow}
Control flow statements govern the sequence of execution in a Java program.

\subsection{if-else Statements}
The if-else statement provides a mechanism for making decisions in your code based on specified conditions, allowing for branching and alternate execution paths.

\subsection{Switch Statements}
Switch statements offer an alternative approach to decision-making, relying on the value of an expression to determine the appropriate course of action.

\section{Loops}
Loops facilitate the repeated execution of a block of code, enabling efficient handling of tasks that involve iteration.

\subsection{for Loop}
The for loop is utilized to iterate over a specified range of values, making it a powerful construct for repetitive tasks.

\subsection{while Loop}
The while loop repeats a block of code as long as a specified condition remains true, providing flexibility in handling varying situations.

\subsection{do-while Loop}
The do-while loop is similar to the while loop but ensures that the block of code is executed at least once, offering specific advantages in certain scenarios.

\section{Methods}
Methods in Java encapsulate a set of instructions, promoting code modularity, reusability, and maintainability.

\subsection{Defining Methods}
Learn how to define methods to encapsulate functionality and promote code organization.

\subsection{Calling Methods}
Understand the process of calling methods, allowing you to leverage existing functionality in your programs.

\subsection{Method Overloading}
Explore the concept of method overloading, enabling the definition of multiple methods with the same name but different parameters.

\section{Object-Oriented Programming (OOP)}
Object-oriented programming is a paradigm that organizes code around objects, fostering modularity, encapsulation, inheritance, and polymorphism.

\subsection{Classes and Objects}
Understand the concept of classes and objects as the core building blocks in object-oriented programming.

\subsection{Inheritance}
Explore inheritance, a mechanism that allows a class to inherit properties and behaviors from another class, promoting code reuse.

\subsection{Polymorphism}
Learn about polymorphism, which allows objects to be treated as instances of their parent class, facilitating flexibility and extensibility.

\subsection{Encapsulation}
Discover the benefits of encapsulation in OOP, where data and methods are bundled together, providing better control and security.

\end{document}