\documentclass{article}
\usepackage[utf8]{inputenc}
\usepackage{fancyhdr}
\usepackage{hyperref} % For hyperlink
\pagestyle{fancy}

\lhead{Chapter 1}
\chead{Nishat Tasnim}
\rhead{\thepage}

\renewcommand{\headrulewidth}{0pt}

\lfoot{Python Book}
\cfoot{\href{www.python.com}{Python}}
\rfoot{Mastering Python}

\fancypagestyle{style}{\fancyhf{}
\fancyhead[l]{Author: Mark Lutz}\fancyhead[c]{Book Title: Learning Python}\fancyhead[r]{Page \thepage}
\fancyfoot[l]{Python Documentation}\fancyfoot[c]{\href{www.python.com}{Python}}\fancyfoot[r]{Mastering Python}}

\begin{document}

\section{Classes and Objects}
\begin{itemize}
    \item In Python, a class is a blueprint for creating objects.
    \item Objects are instances of classes.
    \item Classes encapsulate data (attributes) and behavior (methods) into a single unit.
\end{itemize}

\section{Encapsulation}
\begin{itemize}
    \item Encapsulation is the bundling of data and methods that operate on the data into a single unit (class).
    \item It allows the hiding of internal details of the class and provides a clear interface for interacting with objects.
\end{itemize}

\section{Inheritance}
\begin{itemize}
    \item Inheritance allows a class (subclass or derived class) to inherit properties and methods from another class (base class or parent class).
    \item It promotes code reusability and the creation of a hierarchy of classes.
\end{itemize}

\section{Polymorphism}
\begin{itemize}
    \item Polymorphism enables objects of different classes to be treated as objects of a common base class.
    \item It allows methods to be written to handle objects of a certain class and its subclasses, promoting flexibility in code design.
\end{itemize}

\section{Abstraction}
\begin{itemize}
    \item Abstraction involves simplifying complex systems by modeling classes based on the essential properties and behaviors they share.
    \item It hides the complex reality while exposing only the relevant functionalities.
\end{itemize}

\section{Method Overriding}
\begin{itemize}
    \item Method overriding allows a subclass to provide a specific implementation of a method that is already defined in its superclass.
    \item It is a way to customize or extend the behavior of the inherited method.
\end{itemize}

\section{Class and Instance Variables}
\begin{itemize}
    \item Class variables are shared among all instances of a class.
    \item Instance variables are unique to each instance.
    \item Class variables are defined outside methods, and instance variables are defined within methods using the \texttt{self} keyword.
\end{itemize}

\section{Object Initialization}
\begin{itemize}
    \item The \texttt{\_\_init\_\_} method is a special method in Python classes used for initializing object attributes when an object is created.
\end{itemize}

\section{Composition}
\begin{itemize}
    \item Composition involves creating complex objects by combining simpler ones.
    \item It is an alternative to inheritance and allows for more flexible relationships between classes.
\end{itemize}

\section{Decorator Pattern}
\begin{itemize}
    \item Python supports the decorator pattern, allowing the modification of the behavior of functions or methods using decorators.
    \item Decorators are a concise way to apply reusable and modular changes to functions or methods.
\end{itemize}In summary, Object-Oriented Programming in Python provides a powerful and flexible way to structure code, promote reusability, and model real-world entities in a software system. The principles of OOP enhance code organization, maintenance, and scalability.


\newpage
\thispagestyle{empty}
\section{Exploring Advanced Concepts}

\subsection{Memory Management in Python}

Delve into Python's memory management mechanisms, including reference counting and garbage collection.
Discuss strategies for optimizing memory usage and avoiding memory leaks.
Explore tools for profiling memory usage and identifying potential bottlenecks.

\subsection{Metaprogramming with Decorators}

Introduce decorators as a powerful tool for modifying function behavior at runtime.
Demonstrate practical examples of decorators for logging, timing, caching, and access control.
Explore advanced techniques like decorator factories and stacking decorators.

\subsection{Concurrency and Parallelism}

Explore concepts of concurrency and parallelism in Python.
Discuss the Global Interpreter Lock (GIL) and its impact on multithreading.
Introduce libraries like asyncio and multiprocessing for concurrent and parallel programming.

\subsection{Context Managers and the \texttt{with} Statement}

Understand the role of context managers in Python.
Learn about the \texttt{with} statement and how it simplifies resource management.
Explore creating custom context managers for specific use cases.


\newpage
\thispagestyle{style}
\section{Mastering Python Techniques}

\subsection{List Comprehensions and Generator Expressions}

Unleash the concise power of list comprehensions and generator expressions for elegant data manipulation.
Explore use cases for creating lists, filtering data, transforming elements, and performing complex operations in a single line.
Compare performance implications and memory efficiency considerations.

\subsection{Effective Error Handling with Exceptions}

Master the art of handling errors gracefully using try-except blocks and exception hierarchies.
Define custom exceptions to signal specific error conditions in your code.
Practice common error handling scenarios, including file I/O, network operations, and user input validation.

\subsection{Object-Oriented Design Principles}

Delve into essential principles of object-oriented design (OOD) in Python.
Explore concepts like encapsulation, inheritance, polymorphism, and abstraction.
Learn how to apply OOD principles to create modular and maintainable code.

\subsection{Web Scraping with Beautiful Soup and Requests}

Introduce web scraping using the Beautiful Soup and Requests libraries.
Learn how to extract data from HTML and parse it for various applications.
Explore ethical considerations and best practices in web scraping.

\end{document}