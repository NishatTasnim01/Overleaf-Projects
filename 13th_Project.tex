\documentclass{article}
\usepackage{graphicx} % Required for inserting images
\usepackage{amsmath}
\usepackage[table,xcdraw]{xcolor}
\usepackage{colortbl}

\title{Thirteenth Class}
\author{Nishat Tasnim}
\date{December 2023}

\begin{document}

\section{Advanced Mathematical Expressions}

\subsection{Fraction and Limit}
\begin{align*}
    \frac{a+b}{c}\\
    \int_{0}^{5}\displaystyle \lim_{{1 \to 5}}\\\\
    \frac{2x+1}{x-3}\\\\
    \int_{-\infty}^{\infty}\displaystyle \lim_{{n \to \infty}}\left(1 + \frac{1}{n}\right)^n\\\\
    \frac{\sin^2(x)}{\cos(x)}\\\\
    \lim_{{x \to 0}}\frac{\tan(x)}{x}\\\\
    \frac{\sqrt{a^2 + b^2}}{a+b}\\\\
    \lim_{{n \to \infty}}\left(1 + \frac{1}{n}\right)^{2n}\\\\
    \frac{e^{2x} - 1}{e^x + 1}\\\\
    \int_{-\pi}^{\pi}\displaystyle \lim_{{h \to 0}}\frac{\sin(x+h) - \sin(x)}{h}
\end{align*}

\subsection{Greek Letters}
\begin{align*}
    \alpha \\
    \beta \\
    \zeta \\
    \kappa \\
    \epsilon \\
    \varepsilon \\
    \iota \\
    \xi \\
    \Gamma \\
    \Delta \\
    \Theta \\
    \Lambda \\
    \theta \\
    \vartheta \\
    \nu \\
    \varrho \\
    \lambda \\
    \mu \\
    \rho \\
    \upsilon \\
    \sigma \\
    \varsigma \\
    \chi \\
    \phi \\
    \varphi \\
    \delta \\
    \omega \\
    \gamma \\
    \eta 
\end{align*}

\section{Mathematical Concepts:}

\subsection{Trigonometry}

\subsubsection{Sine Function}
The sine function, denoted as \(\sin(\theta)\), is a trigonometric function that relates the angle \(\theta\) of a right triangle to the ratio of the length of the side opposite to \(\theta\) to the length of the hypotenuse.

\subsubsection{Cosine Function}
The cosine function, denoted as \(\cos(\theta)\), is a trigonometric function that relates the angle \(\theta\) of a right triangle to the ratio of the length of the adjacent side to \(\theta\) to the length of the hypotenuse.

\subsection{Calculus}

\subsubsection{Derivative}
The derivative of a function \(f(x)\) at a point \(x=a\) is defined as the limit of the difference quotient as \(h\) approaches 0:
\[ f'(a) = \lim_{{h \to 0}} \frac{f(a + h) - f(a)}{h} \]

\subsubsection{Integral}
The integral of a function \(f(x)\) over an interval \([a, b]\) is the limit of Riemann sums as the partition of the interval approaches zero:
\[ \int_{a}^{b} f(x) \,dx = \lim_{{\Delta x \to 0}} \sum_{{i=1}}^{n} f(x_i) \Delta x \]

\newpage
\section{Matrix:}
\begin{equation}
    \begin{bmatrix}
    1 & 2 & 3 & 4 \\
    5 & 6 & 7 & 8 \\
    9 & 10 & 11 & 11 \\
    12 & 13 & 14 & 15 \\
    \end{bmatrix}
\end{equation}\\

\section{Colorful Table:}
\begin{table}[h]
\begin{tabular}{|cc|cc|}
\hline
\rowcolor[HTML]{EFEFEF} 
\multicolumn{2}{|c|}{\cellcolor[HTML]{EFEFEF}{\color[HTML]{010066} \textbf{Previous Data}}}                           & \multicolumn{2}{c|}{\cellcolor[HTML]{EFEFEF}{\color[HTML]{010066} \textbf{Subsequent Data}}}                         \\ \hline
\rowcolor[HTML]{010066} 
\multicolumn{1}{|c|}{\cellcolor[HTML]{010066}{\color[HTML]{FFFFFF} \textbf{5}}}  & {\color[HTML]{FFFFFF} \textbf{6}}  & \multicolumn{1}{c|}{\cellcolor[HTML]{010066}{\color[HTML]{FFFFFF} \textbf{7}}}  & {\color[HTML]{FFFFFF} \textbf{8}}  \\ \hline
\rowcolor[HTML]{EFEFEF} 
\multicolumn{1}{|c|}{\cellcolor[HTML]{EFEFEF}{\color[HTML]{010066} \textbf{9}}}  & {\color[HTML]{010066} \textbf{10}} & \multicolumn{1}{c|}{\cellcolor[HTML]{EFEFEF}{\color[HTML]{010066} \textbf{11}}} & {\color[HTML]{010066} \textbf{12}} \\ \hline
\rowcolor[HTML]{010066} 
\multicolumn{1}{|c|}{\cellcolor[HTML]{010066}{\color[HTML]{FFFFFF} \textbf{13}}} & {\color[HTML]{FFFFFF} \textbf{14}} & \multicolumn{1}{c|}{\cellcolor[HTML]{010066}{\color[HTML]{FFFFFF} \textbf{15}}} & {\color[HTML]{FFFFFF} \textbf{16}} \\ \hline
\end{tabular}
\end{table}

\begin{table}[h]
\begin{tabular}{|cc|cc|}
\hline
\rowcolor[HTML]{E4EBF5} 
\multicolumn{2}{|c|}{\cellcolor[HTML]{E4EBF5}{\color[HTML]{360C63} \textbf{Previous Data}}}                           & \multicolumn{2}{c|}{\cellcolor[HTML]{E4EBF5}{\color[HTML]{360C63} \textbf{Subsequent Data}}}                         \\ \hline
\rowcolor[HTML]{D3B1D1} 
\multicolumn{1}{|c|}{\cellcolor[HTML]{D3B1D1}{\color[HTML]{360C63} \textbf{5}}}  & {\color[HTML]{360C63} \textbf{6}}  & \multicolumn{1}{c|}{\cellcolor[HTML]{D3B1D1}{\color[HTML]{360C63} \textbf{7}}}  & {\color[HTML]{360C63} \textbf{8}}  \\ \hline
\rowcolor[HTML]{E4EBF5} 
\multicolumn{1}{|c|}{\cellcolor[HTML]{E4EBF5}{\color[HTML]{010066} \textbf{9}}}  & {\color[HTML]{010066} \textbf{10}} & \multicolumn{1}{c|}{\cellcolor[HTML]{E4EBF5}{\color[HTML]{010066} \textbf{11}}} & {\color[HTML]{010066} \textbf{12}} \\ \hline
\rowcolor[HTML]{D3B1D1} 
\multicolumn{1}{|c|}{\cellcolor[HTML]{D3B1D1}{\color[HTML]{360C63} \textbf{13}}} & {\color[HTML]{360C63} \textbf{14}} & \multicolumn{1}{c|}{\cellcolor[HTML]{D3B1D1}{\color[HTML]{360C63} \textbf{15}}} & {\color[HTML]{360C63} \textbf{16}} \\ \hline
\rowcolor[HTML]{E4EBF5} 
\multicolumn{1}{|c|}{\cellcolor[HTML]{E4EBF5}{\color[HTML]{360C63} \textbf{17}}} & {\color[HTML]{360C63} \textbf{18}} & \multicolumn{1}{c|}{\cellcolor[HTML]{E4EBF5}{\color[HTML]{360C63} \textbf{19}}} & {\color[HTML]{360C63} \textbf{20}} \\ \hline
\end{tabular}
\end{table}

\end{document}