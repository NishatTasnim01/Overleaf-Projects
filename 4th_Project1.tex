\documentclass{article}
\usepackage{fontspec}
\usepackage{polyglossia}
\setmainlanguage{english}
\setotherlanguage{bengali}

\newfontfamily\bengalifont[Script=Bengali]{Kalpurush} 

\title{Bangla Font}
\author{Nishat Tasnim}
\date{December 2023}

\begin{document}

\maketitle


\section{\texorpdfstring{\begin{bengali}বাংলাদেশের জাতীয় সঙ্গীত আমার সোনার বাংলা\end{bengali}}{Bengali National Music My Golden Bengal}}
\begin{bengali}
আমার সোনার বাংলা, আমি তোমায় ভালোবাসি।\\
চিরদিন তোমার আকাশ, তোমার বাতাস, আমার প্রাণে বাজায় বাঁশি॥\\
ও মা, ফাগুনে তোর আমের বনে ঘ্রাণে পাগল করে,\\
মরি হায়, হায় রে—\\
ও মা, অঘ্রানে তোর ভরা ক্ষেতে আমি কী দেখেছি মধুর হাসি॥\\
\\কী শোভা, কী ছায়া গো, কী স্নেহ, কী মায়া গো—\\
কী আঁচল বিছায়েছ বটের মূলে, নদীর কূলে কূলে।\\
মা, তোর মুখের বাণী আমার কানে লাগে সুধার মতো,\\
মরি হায়, হায় রে—\\
মা, তোর বদনখানি মলিন হলে, ও মা, আমি নয়নজলে ভাসি॥\\
\\তোমার এই খেলাঘরে শিশুকাল কাটিলে রে,
তোমারি ধুলামাটি অঙ্গে মাখি ধন্য জীবন মানি।\\
তুই দিন ফুরালে সন্ধ্যাকালে কী দীপ জ্বালিস ঘরে,\\
মরি হায়, হায় রে—\\
তখন খেলাধুলা সকল ফেলে, ও মা, তোমার কোলে ছুটে আসি॥\\
\\ধেনু-চরা তোমার মাঠে, পারে যাবার খেয়াঘাটে,\\
সারা দিন পাখি-ডাকা ছায়ায়-ঢাকা তোমার পল্লীবাটে,\\
তোমার ধানে-ভরা আঙিনাতে জীবনের দিন কাটে,\\
মরি হায়, হায় রে—\\
ও মা, আমার যে ভাই তারা সবাই, ও মা, তোমার রাখাল তোমার চাষি॥\\
\end{bengali}

\subsection{\texorpdfstring{\begin{bengali}তথ্যকণিকা\end{bengali}}{Data Cell}}
\begin{bengali}
আমার সোনার বাংলা হলো বাংলাদেশের জাতীয় সঙ্গীত। বঙ্গমাতা সম্পর্কে এই গানটি রবীন্দ্রনাথ ঠাকুর কর্তৃক ১৯০৫ সালে রচিত হয়। বাউল গায়ক গগন হরকরার গান "আমি কোথায় পাব তারে" থেকে এই গানের সুর ও সঙ্গীত উদ্ভূত।
\end{bengali}

\section{\texorpdfstring{\begin{bengali}একুশের গান\end{bengali}}{The Song of Twentyfirst}}
\begin{bengali}
আমার ভাইয়ের রক্তে রাঙানো একুশে ফেব্রুয়ারি\\
আমি কি ভুলিতে পারি\\
ছেলেহারা শত মায়ের অশ্রু গড়া-এ ফেব্রুয়ারি\\
আমি কি ভুলিতে পারি\\
আমার সোনার দেশের রক্তে রাঙানো একুশে ফেব্রুয়ারি\\
আমি কি ভুলিতে পারি।।\\
\end{bengali}

\subsection{\texorpdfstring{\begin{bengali}তথ্যকণিকা\end{bengali}}{Data Cell}}
\begin{bengali}
একুশের গান একটি বাংলা গান যাআমার ভাইয়ের রক্তে রাঙানো হিসেবে সুপরিচিত (প্রথম চরণ দ্বারা)। এই গানের কথায় ১৯৫২ সালের ফেব্রুয়ারি ২১ তারিখে সংঘটিত বাংলা ভাষা আন্দোলনের ইতিহাস ফুটে উঠেছে। সাংবাদিক ও লেখক আবদুল গাফফার চৌধুরী ১৯৫২ সালের ২১শে ফেব্রুয়ারিতে গানটি রচনা করেন।
\end{bengali}

\subsection{\texorpdfstring{\begin{bengali}রচয়িতা\end{bengali}}{Author}}
\begin{bengali}
আবদুল গাফফার চৌধুরী (১২ ডিসেম্বর ১৯৩৪ — ১৯ মে ২০২২) ছিলেন একজন বাংলাদেশী গ্রন্থকার, কলাম লেখক। তিনি ভাষা আন্দোলনের স্মরণীয় গান আমার ভাইয়ের রক্তে রাঙানো-এর রচয়িতা।
\end{bengali}

\section{\texorpdfstring{\begin{bengali}নতুনের গান\end{bengali}}{The Song of Youth}}
\begin{bengali}
চল্ চল্ চল্। চল্ চল্ চল্।\\
ঊর্ধ্ব গগনে বাজে মাদল\\
নিম্নে উতলা ধরণী-তল\\
অরুণ প্রাতের তরুণ দল\\
চল্ রে চল্ রে চল্\\
চল্ চল্ চল্।।\\
\end{bengali}

\subsection{\texorpdfstring{\begin{bengali}তথ্যকণিকা\end{bengali}}{Data Cell}}
\begin{bengali}
নতুনের গান বাংলাদেশের জাতীয় কবি কাজী নজরুল ইসলাম কর্তৃক ১৯২৯ খ্রিস্টাব্দে রচিত এবং সুরারোপিত সন্ধ্যা কাব্যগ্রন্থের অন্তর্গত একটি গান। দাদরা তালের এই সঙ্গীতটি ১৯৭২ সালের ১৩ই জানুয়ারি অনুষ্ঠিত বাংলাদেশের তৎকালীন মন্ত্রীসভার প্রথম বৈঠকে বাংলাদেশের রণ-সঙ্গীত হিসেবে নির্বাচন করা হয়।
\end{bengali}

\subsection{\texorpdfstring{\begin{bengali}কবি\end{bengali}}{Poet}}
\begin{bengali}
কাজী নজরুল ইসলাম (২৪ মে ১৮৯৯ – ২৯ আগস্ট ১৯৭৬; ১১ জ্যৈষ্ঠ ১৩০৬ – ১২ ভাদ্র ১৩৮৩ বঙ্গাব্দ) বিংশ শতাব্দীর প্রধান বাঙালি কবি ও সঙ্গীতকার। তার মাত্র ২৩ বছরের সাহিত্যিক জীবনে সৃষ্টির যে প্রাচুর্য তা তুলনারহিত। সাহিত্যের নানা শাখায় বিচরণ করলেও তার প্রধান পরিচয় তিনি কবি।
\end{bengali}

\end{document}